\chapter{Introduction}
\label{chapter:intro}

This master's thesis purpose is to find out how to sell services for limited liability housing companies in a situation where the case company sells the service for another company and then to the end customer. The study uses design science research approach and it is conducted as a service design project.

This chapter begins by presenting the background and motivation for the research and introducing the case company. Then the problem and research questions are described and finally, the structure of the thesis is presented.

\section{Background and motivation}

In Finland about 2,7 million people are living in approximately 88 000 limited liability housing companies \parencite{REMF, Stats}. This limited liability housing company regime is an unique setting compared to other countries and it is dominant form of ownership in multi-apartment buildings in Finland \parencite{Lujanen:2017}. The operation of the limited liability company is regulated in the Finnish law \parencite{YIT}.

There are a lot of books and online material to support and guidance the people who are in the board of a limited liability housing company. In addition, it is discussed a lot that the work in a limited liability housing company can be a hassle and people do not want to participate in the operation of the housing company. However, there is a lack of knowledge and understanding about the buying decision making process as well as the motivation of the people to be in the board.

This study tries to fill the knowledge gap by researching and understanding the people who are working in a limited liability housing company and create a better way to reach the thousands of limited liability housing companies by identifying some similarities between them.

Related to the energy industry this study concentrates, the aim is to facilitate a more customer-centric approach since according to \textcite{Fader:2012} enegy firms do not refer to their customers as \emph{customers} but \emph{rate-payers}. This emphasizes how the energy firms view their customers as homogenous and what could be changed.

\section{Introduction to case company}

The case company in this study is Leanheat which provides a software solution that enables the space  heating optimization of centrally heated multi-family residential buildings. This is done by installing wireless sensors to track temperature and humidity, and a connection to the central heating controller is established. The system collects data from apartments, heating system and weather in order to automatically create a thermodynamic model for the building. Then the space heating is optimized based on the parameters resulting a minimized simultaneous usage of heat and thus lowering the peak loads caused by it. \parencite{LenheatArticle:2019}

The traditional heating system can not adjust to the changing weather or the thermodynamic properties of the building. In addition, the traditional system does not communicate with other systems inside the buildings nor with the buildings within the same network. This causes a suboptimal behaviour in the system, since it possible that all of these components require high heat power at the same time. \parencite{LenheatArticle:2019}

The benefits of the Leanheat's solution for district heating companies are that 1) the peak loads are 20 \% lower and primary return temperatures are two to four degrees lower, 2) heat production is optimized with demand forecasts and demand response and 3) data from the buildings offer new business models to be used and developed. For residents the solution provides stable and better indoor conditions. \parencite{LenheatArticle:2019}

In Finland, from the big apartment buildings around one third is owned by the professional property owners and in that specific market the case company is already the market lead. The rest of the buildings are limited liability housing companies which means that reaching them as clients is strategically important in order to grow bigger in the market.

Leanheat and district heating companies are forming partnerships because the system provides a way for the district heating companies to optimize their energy production. Moreover, based on the research done by \textcite{Energyindustry:2019} customers are becoming active in acquiring services which means that the energy companies should be more active in providing service business in order to engage better with their customers.

\section{Problem statement}

As stated above, acquiring limited liability housing companies as clients is difficult, because there are tens of thousands of them and those are run by laypersons. In addition, most of them operate in "status quo" which means that the goal of the private housing company is to keep the building as it is rather than developing it. Selling services for limited liability housing companies requires resources and knowledge about the end customer. On top of that, district heating companies are valuable since currently they have their exiting customer relationships with limited liability housing companies because they provide them heat.

The research will concentrate on creating new and better understanding on the buying decision making processes inside limited liability housing companies by exploring the background motivation for people to be in the board and what affects the decision making. With that understanding it is then possible to design and develop a sales process and guidelines on how to reach the potential market of the limited liability housing companies. Thus, the first research question is formed as follows:\\\\
\emph{RQ1: What affects the purchasing decision making process in a limited liability housing companies?}\\

In addition, the partnerships should be taken into account when designing and developing in order to find out the value for district heating companies. By doing this research the aim is to create a new service business opportunity and by concrete increase sales, reduce lead times and customer acquisition costs. To answer this, the second research question is formulated as follows:\\\\
\emph{RQ2: What kind of sales process could help engage the end-customer in B2B2C setting?}\\

\section{Structure of the Thesis}
\label{section:structure} 

In the next chapter, the background information about limited liability housing companies is studied and presented to gain understanding about what they are and how they operate. In addition, Chapter ~\ref{chapter:background} present the environment concerning the energy business and its key features related to service business and future development. The research methods are presented in Chapter ~\ref{chapter:methods} along with the description on how the methods were implemented in the study. In Chapter ~\ref{chapter:results} the results of the study are presented. Finally, the discussion related to the results of the study as well as limitations of the study and future research are discussed in the Chapter ~\ref{chapter:discussion}.