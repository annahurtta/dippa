\chapter{Discussion}
\label{chapter:discussion}

This chapter begins by presenting the answers to the research questions of the study. Then, the limitations of the study are gone through. Lastly, the recommendations of future research as well actions are presented.

\section{Answers to the Research Questions}

In this chapter the answers to the research question are presented and reflected to the previous literature. In addition to the answers for the research questions, the researcher suggest actions.

\subsection{RQ1: Purchase Decision Making in Limited Liability Housing Companies}

The first research question asked about what affects the purchase decision making in a limited liability housing company. More specifically, the study investigated what kind of people there are in the boards of housing companies and how they operate in order to understand the motives behind their decision making. In chapter 4, the factors influencing on the purchase decisions were described through themes found in the interviews which were motivation of the board, role of the deputy landlord, decision making in general and board culture.

Based on the interviews, \textbf{one of the biggest factor which affects the purchase decision making is the people who are in the board} since they are influencing the actions and operation of the housing company based on their own background and interest. Issues related to the original thought that the decision making is hard, found out to be wrong and the issue is more about the lack of understanding the topic being decided about. This finding is in line with the dissertation \emph{Uncertainty in Consumer Online Search and Purchase Decision Making} made by \textcite{PurchaseDecisionMaking:2011} where it is referenced that there are two types of uncertainty in purchase decision making, knowledge uncertainty and choice uncertainty. \textcite{PurchaseDecisionMaking:2011} writes that knowledge uncertainty means the doubts the consumers have related to their own ability to judge the service providers as well as the product itself which makes the decision making more complicated since it is then hard to do evaluate the options and it drives consumers to do more pre-purchase search. Based on the interview insights the knowledge uncertainty is affecting also the board members ability to make decisions.

\subsubsection*{Motivation and Activity of the Board}

The motivation of the board should be considered from two perspectives. First, the motivation is highly dependent on the people who are in the board and secondly, the motivation and aspirations of the board is driving the actions of the board and thus it affects the decision making. As mentioned previously, if the board do not have anyone who would have professional experience for example from the real estate industry, it is highly likely that the deputy landlord is then the one who is leading the housing company actions. This factor was also noticed in the thesis made by \textcite{PehkonenThesis:2012} where she researched energy efficiency related thoughts in housing companies in the area of Helsinki. \textcite{PehkonenThesis:2012} states in the thesis that the housing companies who have professional experience or an active deputy landlord stand out from the group with activity.

\subsubsection*{Board Members' Problems and Needs}

The problems and needs identified in housing companies related to their operations and as well as purchase decision making can be formed from the interviews and are as follow:
\begin{itemize}
	% You can use this command to set the items in the list closer to each other
	% (ITEM SEParation, the vertical space between the list items) 
	\setlength{\itemsep}{1pt}
	\item Deputy landlord do not always meet the needs of a housing company.
	\item Lack of workmanship inside the board leads to a situation where the goal is to keep the building as it is currently.
	\item Lack of knowledge and understanding may lead to a situation where no decision is done.
	\item Recruiting new people to the board is considered as hard and people do not have interest towards the housing company actions enough.
	\item Board needs professional consulting to make decision if they do not have knowledge related to the topic or they do not have the knowledge inside the board.
\end{itemize}

In conclusion the operation of the housing board is dependent of the people and their background. In addition, the housing companies are dependent on the deputy landlord for actually running the operations. The more active the board is the less important role deputy landlord usually has. In both cases, the deputy landlord's suggestions are heard. What is important to notice is the fact that the decisions are made firstly in order to provide good living conditions for residents and secondly to provide them in cost effective way. The decision making in a housing company should be considered as a whole and it is important to identify what type of a board it is in order to understand their motives behind the operation.

The interviews also indicate that the service provided by the case company is not applicable as it is currently for the housing companies because it is designed for professional usage. Thus, the communication about the service as well as the product itself need to be simplified and re-though for the housing companies since for them the concept of energy efficiency is not that well understood and it might also mean worse indoor conditions. The fear of receiving lower indoor temperatures as a result of energy efficiency was also stated by \textcite{PehkonenThesis:2012}.

\subsection{RQ2: Improving the Sales Process}

The second research question focused on finding out what kind of sales process would help energy companies to sell services for housing companies. To answer this question the current sales process was needed to go through with customer journey mapping and then do small tests related to different phases of the journey which were identified as bottlenecks.

\subsubsection*{Limited Liability Housing Company is not a Business Customer}

Based on the research one of the most important aspect of selling services to housing company is to internalize the fact that housing company is not a business customer even though the name says limited liability company. The board of the housing company is consisting of so to say normal people who are mainly working in the board voluntarily even though they might receive a small compensation for the effort. Moreover, the energy companies should see the housing companies as actual customers and not as rate-payers, in order to understand them and to provide them services which they need.

\subsubsection*{Providing Guidance for the Board Members}

What it comes to the presentation of the provided services, there are few things which should be considered in the housing company scene. Since, one person in the board can not make decisions alone, the whole presentation needs to be considered from the perspective that it also provides guidance for the one who is presenting the service for other board members. Here are listed the factors which should be in mind while designing the actual sales material. \emph{The user needs a way to:}
\begin{itemize}
\item Understand what the company is selling and why it is important.
\item Present the product for other board members.
\item Make the purchase decision by consulting someone professional if (s)he doesn't have the skills related to the topic.
\item Know if the provided service is needed/relevant for the building.
\end{itemize}

\subsubsection*{Service Fee Based Pricing Model}

Related to the pricing of the service, it should be either service fee based. Even though, the research indicate that the service fee is the most wanted among the housing company board, some housing companies might have more asset and are capable to invest more money in the beginning to gain more benefit in the long run. As mentioned earlier in Background chapter \textcite{Decoded:2013} states the price perception is dependent on many things, so one way to make purchase decision easier could be to show the estimated savings, if there service's value proposition is offering energy savings. However, if the value proposition is about the savings the company need to get a certain amount of savings to believe the value. If the savings are not perceived to be enough, the value proposition can backfire because of the fear of having worse indoor conditions. Thus, based on \textcite{Decoded:2013}, the purchase decision might not happen since the reward do not exceed the pain.

\subsubsection*{Value Proposition}

If the value of the service is thought more from the perspective of the identified board types, those could be aimed to fill their specific needs. For the trend setters the value of the service could be to have a new way of heating, since the current one is outdated. For guardians, the value of the service could be about taking care of the investment's value and to offer a tool which can help make renovations in advanced. Lastly, for the responsibles the value could be to provide a tool which alert about possible problems before the big renovation is needed.

\subsubsection*{Target the Trendsetters}

In the context of selling a service which is totally new in the market the boards who are trendsetters, as presented in the chapter 4, should be identified. Since, they are the most proactive it could be also the easiest way to sell them the service first. Through this then, the care takers and responsible would follow the trendsetters. One of the issues in selling something new is that people do not have knowledge about it, so it is hard to start even looking for it. In this situation, the responsibility to provide such information lies on the service provider. This suggestion is also supported in the literature. According to \textcite{PurchaseDecisionMaking:2011} expert consumers do not need information and they do not search it and novice users do not search information because they lack the ability to do so. Moreover, as \textcite{Sinek:2009} stated, to succeed in the mass-markets the company first need to get 15-18 \% of the market and this can not happen without the early adopters.

\subsubsection*{Role of the Deputy Landlords}

Based on the interviews and from the testing, it came clear that the limited deputy landlord is the trusted person among the board of housing companies and that they need consulting as well as support for decision making if the matter is complicated or rather new. Since, one deputy landlord handles usually many housing companies at once, they reach many potential customers and have relationship with them already. Thus, deputy landlord could be the one to suggest new services for housing company if they believe and see value in the service themselves. If this is the case, it is important to remember that deputy landlords want to stay as independent and they can not nor want to take commission from recommending the solution. In addition, it should be taken into account, based on the interviews, that not all deputy landlords are meeting the expectations of the boards. In this case, it is important to find the deputy landlords who are active and have the correct mindset about being service provider for the housing companies. Deputy landlords also know their customers, so with their help it would be easier to identify the boards who are classified as trendsetters.

\subsubsection*{Customer Centricity into Sales Process}

To summarize, the energy companies should re-think their way of selling to be more customer-centric in a way that it would answer the needs of the housing companies. It would be beneficial to arrange events for housing companies, where they would get knowledge about the service in a way that is presented clearly and relatable since it is problematic when the board members receives an offer and the information about the service at once. In addition, the path-to-purchase should be as easy and effortless as possible, sine the board members are not being too active based on the interviews. Because, the interviewees stated that their responsibility is to provide good living conditions to all residents the main value proposition could be about providing stable, equal and healthier indoor conditions to all residents. Deputy landlord should be involved in the process, since they know their customers and they are the trusted person with expertise inside the board.

\section{Limitations of the study}

The study has limitations which should be acknowledged. Related to the data collection process, the researched did the interviews alone and it is possible that even though the interview questions were open ended the interviewer directed the answers of the interviewee. The generalisation of the interview results is geographically uncertain since the interviewees all lived in the capital area of Finland, still the interviewees represented well different age groups, different levels of experience in the board as well as gender.

In addition, it is possible that the interviewees' answers were affected by the desire to please the interviewer. Nevertheless, the interviewees talked openly about their feelings and motivations related to working in the housing board even though in some cases they were negative and critical, which indicate that the answers represent their opinions in real life. 

During the data analyze phase only one researcher analyzed the interviews and because of that the previous knowledge and perspectives could have affected how the researcher analyzed and categorized the interviews. To be noted, the researched did not have any experience on being part of the housing board so she did not have any strong opinions or own experiences that could have highly affected the analysis.

Because of the time span of the thesis, the researcher did not have enough time to iterate the tests according to the design science research, so the suggestions would need more testing to be generalized in a larger context. In addition, it is hard to measure if the research results helped to reduce customer acquisition costs, reduce lead times and increase sales since the test were concentrating on small parts of the whole customer journey compared to contemplating it as a whole.

\section{Impact of the Research}

The research had impact in the case company. Firstly, the form of communication in ads, offers and presentations where the focus was on the professional property owner, transformed to focus on the people who are living at their own homes. This means in practice for example that the value proposition is focusing more on providing the good living conditions rather than saving energy. Additionally, the role of deputy landlords is increased and their importance acting as the ones who are suggesting the solution is better understood. In practice this means that in some cases the deputy landlords are the ones who enable the sales to happen. Moreover, the mindset where the housing companies are seen as business customer is slowly changing across the company. All in all, the research affected the daily work of the team where the researcher worked during the study and the conducted interviews provided a lot of new knowledge about the housing companies.

\section{Future work}

The future work and research could be about researching more about issues and needs that the housing companies has related to heating and energy efficiency. Since, there are already few heating optimization services in the market it could be beneficial to do user research with the ones who are already using the service. Through the research, the company would get more information on what is really important for the housing company and what aspects of the solution brings them value and benefits.

In addition, the future work could continue to test the researcher's suggestions as a complete customer journey experience with one energy company. The future work should concentrate on finding out how the deputy landlords would use the service and how the whole maintenance chain could be designed around the service in order to be fully predicting and offering added value for the housing companies.