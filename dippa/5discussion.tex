\chapter{Discussion}
\label{chapter:discussion}

This chapter consists of the discussion about the results related to the research questions. In addition, the limitations of the study are gone through. Lastly the recommendations of future research as well actions are presented.

\section{Answers to the Research Question}

In this chapter the answers to the research question are presented and reflected to the previous literature. In addition to the answers for the research questions, the researcher suggest actions to do related to them.

\subsection{Buying Decision Making in Limited Liability Housing Company (RQ 1)}

The first research question asked about the factors which affect the buying decision making in a limited liability housing company. More specifically, the study investigated what kind of people there are in the boards of limited liability housing companies and how they operate in order to understand the motives to make decisions. In chapter 4, the factors influencing on the buying decision making were described through themes found in the interviews which were motivation of the board, role of the deputy landlord, decision making in general and board culture.

Based on the interviews, one of the biggest findings was the fact that the people who are in the board are influencing the actions and operation of the limited liability housing company the most. Issues related to the original thought that the decision making is hard found out to be more about understanding the topic being decided about rather than the decision making itself. This finding is in line with the dissertation \emph{Uncertainty in Consumer Online Search and Purchase Decision Making} made by \textcite{PurchaseDecisionMaking:2011} where it is referenced that there are two types of uncertainty in purchase decision making, knowledge uncertainty and choice uncertainty. Knowledge uncertainty means the doubts the consumers have related to their own ability to judge the service providers as well as the product itself which makes the decision making more complicated since it is then hard to do evaluate the options and it drives consumers to do more pre-purchase search. Based on the interviews it can be stated that the actual buying decision making is pretty simple once the issue being decided is made clear for everyone and there is no knowledge uncertainty.

As stated also in the chapter 2.1, the board of the limited liability housing company can make decision by themselves also without the shareholder's meeting. These decisions are usually budgeted and do not contain any big investments or they do not affect the shareholder's possessed property usage. It usually depends on the board that how and which of the decisions are presented to the shareholder's meeting. Albeit, there are situations where not that many people attend the shareholder's meeting, it is crucial to provide enough information for all in order to reach the wanted outcome.

The motivation of the board should be considered from two perspectives. First, the motivation is highly dependent on the people who are in the board and secondly, the motivation and aspirations of the board is driving the actions of the board. As mentioned previously, if the board do not have anyone who would have professional experience for example from the real estate industry, it is highly likely that the deputy landlord is then the one who is leading the limited liability housing company actions. This factor was also noticed in the thesis made by \textcite{PehkonenThesis:2012} where she researched energy efficiency related thoughts in limited liability housing companies in the area of Helsinki. Pehkonen states in the thesis that the limited liability housing companies who have professional experience or an active deputy landlord stand out from the group with activity.

The problems and needs identified in limited liability housing companies related to their operations and making buying decisions can be formed from the interviews and are as follow:
\begin{itemize}
	% You can use this command to set the items in the list closer to each other
	% (ITEM SEParation, the vertical space between the list items) 
	\setlength{\itemsep}{1pt}
	\item Deputy landlord do not always meet the needs of a housing company.
	\item Lack of workmanship inside the board leads to a situation where the goal is to keep the building as it is currently.
	\item Lack of knowledge and understanding may lead to decision making where everyone do not know what they are actually deciding.
	\item Recruiting new people to the board is hard in some cases and people do not have interest towards the housing company actions enough.
	\item Board needs professional consulting to make decision if they do not have knowledge related to the topic or they do not have the knowledge inside the board.
\end{itemize}

In conclusion the operation of the housing board is dependent of the people and their background. In addition, the housing companies are dependent on the deputy landlord for actually running the operations. The more active the board is the less important role deputy landlord usually has. What is important to notice is the fact that the decisions are made firstly in order to provide good living conditions for residents and secondly to provide them in cost effective way.

The interviews also made clear that the service provided by the case company is not applicable as it is currently for the private housing companies because the lack of knowledge and expertise. Thus, the communication about the service  as well as the product itself need to be simplified and re-though for the private housing companies since for them the concept of energy efficiency is not that well understood and it might also mean worse indoor conditions. The fear of receiving lower indoor temperatures as a result of energy efficiency was also stated by \textcite{PehkonenThesis:2012}. This observation and the interview results were the base for then creating guidance about how to sell services for limited liability housing companies as well as what needs to be considered in that scenario. These suggestions are presented in the next chapter.

\subsection{Guidance About Sales process (RQ 2)}

(discussion ristiriita että isännöitsijät toivoo aktiivisuutta taloyhtiöltä ja päinvastoin, ihmisllä ei tietoa palvelusta joten eivät myöskään osaa etsiä)

(discussion, huono kun pitää mennä tarjous / info yhdessä samassa paketissa, jolloin hirveesti infoa yhdessä lapussa).

The second research question focused on finding out what kind of sales process would help energy companies to sell services for limited liability housing companies. To answer this question the current sales process was needed to go through with customer journey mapping and then do small tests related to different phases of the journey.

Based on the research on of the most important aspect of selling services to limited liability housing company is to internalize the fact that limited liability housing company is not a business customer even though the name says limited liability company. The board of the limited liability housing company is consisting of so to say normal people who are mainly working in the board voluntarily even though they might receive a small compensation for the effort. Moreover, the energy companies should see the limited liability housing companies as actual customers and not as rate-payers, in order to understand them and to provide them services which they need.

What it comes to the presentation of the provided services there are few things which should be considered in the limited liability housing company scene. Since, one person in the board can't make decisions alone the whole presentation needs to be considered from the perspective that it also provides guidance for the one who is presenting the service for other board members. Here are listed the factors which should be in mind while designing the actual sales material. \emph{The user needs a way to:}
\begin{itemize}
\item Understand what the company is selling and why it is important.
\item Present the product for other board members
\item Make the buying decision by consulting someone professional if (s)he doesn't have the skills related to the topic
\item Know if the provided service is needed/relevant for the building
\end{itemize}

Related to the pricing of the service, it should be either service fee based or investment plus service fee. Even though, the research indicate that the service fee is more wanted among the limited liability housing company board, some companies might have more asset and are capable and willing to invest more money in the beginning to gain more benefit in the long run.

Based on the interviews and from the testing, it came clear that the deputy landlord is the trusted person among the board of limited liability housing companies. Moreover, one deputy landlord handles usually many limited liability housing companies at once, meaning that they reach many potential customers and have relationship with them already. Thus, deputy landlord could be a potential way to sell new services for limited liability housing company if they believe and see value in the service themselves. This however is not as easy as it sounds because there was found to be a conflict where some deputy landlords wish limited liability housing companies to be active and vice versa. In this situation it is then hard to identify, who should be the one to make the limited liability housing company operation better.

In the context of selling a service which is totally new in the market the boards who are trendsetters, as presented in the chapter 4, should be identified. Since, they are the most proactive it could be also the easiest way to sell them the service first. Through this then, the care takers and responsible would follow the trendsetters. One of the issues in selling something new is that people do not have knowledge about it, so it is hard to start even looking for it. In this situation, the responsibility to provide such information lies on the service provider. Also according to \textcite{PurchaseDecisionMaking:2011} expert consumers do not need information and they do not search it which is also typical for novice users. However, the reason why novice consumers do not search information is because they lack the ability to do so. Additionally, the consumers who are in the middle are easiest to target since they can be influenced via advertises with the least effort. ( tähän viittaus kirjallisuuteen trend settereistä tai early adoptereista the part of early adapters are a ) 

The analyze of the interviews brought up also the fact that energy efficiency and green values are not seen as important or understood the way professionals do. In addition, the interviewees listed that their responsibility is to provide good living conditions to all residents. Having these facts in mind the current value proposition could be transformed from saving energy in a energy efficient way of heating towards providing stable and equal living conditions to all residents in the apartment building.

\section{Limitations of the study}

The study has limitations which should be acknowledged. Related to the data collection process, the researched did the interviews alone and it is possible that even though the interview questions were open ended the interviewer directed the answers of the interviewee. The generalizability of the interview results is geographically uncertain since the interviewees all lived in the capital area of Finland, still the interviewees represented well different age groups, different levels of experience in the board as well as gender.

In addition, it is possible that the interviewees' answers were affected by the desire to please the interviewer. Nevertheless, the interviewees talked openly about their feelings and motivations related to working in the limited liability housing board even though in some cases they were negative and critical, which indicate that the answers represent their opinions in real life. 

During the data analyze phase only one researcher analyzed the interviews and because of that the previous knowledge and perspectives could have affected how the researcher analyzed and categorized the interviews. To be noted, the researched did not have any experience on being part of the limited liability housing board so she did not have any strong opinions or own experiences that could have highly affected the analysis.

Because of the time span of the thesis, the researcher did not have enough time to iterate the tests according to the design science research, so the suggestions would need more testing to be generalized in larger context. In addition, it is hard to measure if the research results helped to reduce customer acquisition costs, reduce lead times and increase sales since the test were concentrating on small parts of the whole customer journey compared to contemplating it as a whole.

\section{Future work}

The future work and research could be about researching more about issues and needs that the limited liability housing companies has related to heating and energy efficiency. Since, there are already few heating optimization services in the market it could be beneficial to do user research with the ones who are already using the service. Through the researc, the company would get more information on what is really important for the limited liability housing company and what brings them value and benefits.

In addition, the future work should concentrate on finding out how the deputy landlords would use the service and how the whole maintenance chain could be designed around the service in order to be fully predicting.
(discuss tuleva tutkimus oikeasta tuotteesta taloyhtiöille)
mitä ongelmia lämmityksessä
tarkemmin energiatehokkuuteen liittyvää juttuu