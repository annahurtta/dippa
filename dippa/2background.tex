\chapter{Background}
\label{chapter:background} 

In this section, the background information of the research is presented. First, the concept of a limited liability housing company is presented along the information related to its operation. Then, the energy industry in Finland is described by focusing on the digitalisation and service business opportunities. After that, the science behind decision making is presented. Lastly, the background section gives an overview about customer centricity and how service design can be used to bring customers into focus.

\section{Limited Liability Housing Company}

In this chapter the regime of limited liability housing company is presented and the operation of it is described in order to provide background information. First the basic concept is explained, then the responsibilities and operation of the housing company is explored.

Limited liability housing company is a type of limited liability company whose purpose is to own and control one or multiple buildings or a part of it where over the half of the combined floor surface area of the apartment or apartments is designated in the articles of association to be possessed by the shareholders and to be used as residential apartments \parencite{LLHA:2}. To put it simple, the company owns the property and the share capital is divided into shares. For the shareholders, these shares then give a right of possession to a specific apartment. \parencite{Lujanen:2017}. So, when a person buys an apartment in a housing company, she become the shareholder of the bought apartment \parencite{YIT}.

As mentioned already in the introduction, 2.7 million people are living in approximately 88 000 housing companies in Finland which makes it dominant form of ownership in multi-apartment buildings in Finland \parencite{REMF, Stats, Lujanen:2017}. The operation of the limited liability company is regulated in the Finnish law and the articles of association contains the internal law and order of the specific housing company \parencite{YIT}. In practice, the housing company takes care of the real estates and the buildings which it controls and owns. \parencite{LLHA:2}. 

\subsection{The Division of Responsibilities}

housing company is led by the board and the deputy landlord. Their job is to act in a way that advances the company's interest \parencite{LLHA:2}. However, this does not mean that the people who are in the board should have expertise in the real estate field. So, the people working in the board are mostly laypersons which should be considered inside the board as well as among the shareholders. \parencite{Hallintotapa:2017}

The board of the housing company needs to have of three to five people, if the articles of association do not state otherwise. The board needs to have a chairperson who is selected by the board. It is notable that the board members do not need to be shareholders of the housing company which means for example that a person, who lives in a rented apartment in the housing company, can be a member of the board. \parencite{Kotitalolehti:2012}

The highest decision-making body is the shareholder's meeting which is held once a year. In a shareholder's meeting every shareholder has a right to vote. The main responsibility of a shareholder is to decide who will be in the board of the housing company. The Table~\ref{table:responsibilities} presents the actors who form the housing company and their responsibilities in addition to the previously mentioned. \parencite{RantanenViiala:2015} 

\begin{table}
\begin{tabular}{|p{2.5cm}|p{2.5cm}|p{6.4cm}|} 
\hline % The line on top of the table
\textbf{Organ} & \textbf{Chooser}  & \textbf{Responsibilities} \\ 
\hline 
Shareholder's meeting & Formed by all shareholders, not chosen & Shareholder's meeting decide on financial statement, maintenance fee, loans and major overhauls. Shareholder's meeting chooses the board members and gives the freedom of responsibility for the board members and the deputy landlord. In addition they chooses the auditor.\\ 
\hline
Board & Shareholder's meeting & The board chooses the deputy landlord, if the board do not take care of the real estate management by themselves. Board accepts the agreements and oversees the finance. Additionally the board makes a five year plan which covers the needed renovations and repairs. The board also presents the renovations done previously.\\
\hline
Deputy landlord & Board & Deputy landlord's responsibility is to take care of the daily operations of the housing company. Deputy landlord executes the decisions which the board has made. In addition deputy landlord prepares the board meetings and act as a secretary.\\
\hline
Service company & Board & Service company operates under deputy landlord and they do all tasks agreed on the agreements, for example cleaning of the public spaces.\\
\hline
Auditor & Shareholder's meeting & Auditor checks the company's accounts and the legality of the government. Auditor is responsible to report the status for the shareholder's meeting.\\
\hline
\end{tabular} % for really simple tables, you can just use tabular
% You can place the caption either below (like here) or above the table
\caption{Limited liability housing company's organs and their responsibilities. \parencite{RantanenViiala:2015}}
% Place the label just after the caption to make the link work
\label{table:responsibilities}
\end{table} % table makes a floating object with a title

The shareholder's meeting is a place where the shareholders, board members and the deputy landlord meet each other and the meeting provides a good place for discussions and questions \parencite{Hallintotapa:2017}. The decision making in a shareholder's meeting happens usually with the majority decision, which means that the suggestion which has over half of the votes is the final decision. If the voting is related to person, the one who receives most of the votes gets elected. In tie situation the chairperson makes the decision, if the voting is not related to electing a person. \parencite{RantanenViiala:2015}

In the shareholder's meeting the housing board's responsibility is to present the scope of the maintenance actions to be done in the housing company. This scope is then decided in the meeting. \parencite{RantanenViiala:2015} Based on the \textcite{LLHA:2} the board requires shareholder's meeting's decision to proceed with actions which are considering the size and operation of the company unusual or far-reaching, affecting essentially to the use of the shareholder's apartment or affecting essentially to the obligation to pay maintenance fee or to other expenses which are caused by the use of the shareholder's possessed apartment.

The role of the deputy landlord is similar to the managing director in a limited liability company since it is the deputy landlord's responsibility to take care of the daily administration.  Additionally, deputy landlord puts in action the decisions made by the housing board and takes care that the accounting is done properly according to the law. \parencite{Sarekoski:2015}

\section{Energy Industry in Finland}

This chapter focuses on the energy industry in Finland and how the digitalisation is affecting the industry and what it enables. The literature review is based on the consultant's reports made about the current state as well as the future possibilities of the energy industry. The literature about the service business is mainly focused on the general information about the service business since there has not been academic research about the service business in energy industry except for few thesis. This is because the industry is currently moving towards service business \parencite{Energyindustry:2019}. 

In Finland there are around 100 district heating companies which are responsible for heat distribution, sale and production. Most of these companies are owned by the municipalities. \parencite{Energyindustry:2019,Poyry:2018} Based on the statistics made by \textcite{Energyindustrygraphs:2018} 46 \% of apartment buildings are part of the district heating network which means that 2,92 million residents are living in apartments which are heated with district heat. Additionally, according to the \textcite{Energiateollisuus:2018} energy industry is facing a time when customers role is changing towards the direction where they have more opportunities to influence the whole industry.

\subsection{Digitalisation in the Energy Industry}

According to an article which was published in \parencite{Energiauutiset:2016} energy industry is the least developed industry what it comes to the digitalisation. A study was done by \textcite{Deloitte} to find out how digitalisation will change energy industry and what it will allow for the energy companies and for the customers. It was stated that digitalisation is changing the business models and creating new services and products as well as forming new kind of customer experience. In addition, the needs, expectations and hopes of the customers are driving the change in the energy industry. Energy efficiency is increasing and new alternative heating forms are becoming more popular, which are changing the current situation as well. \parencite{Energiateollisuus:2018}

According to the vision statement made by \textcite{Energiateollisuus:2018} Finland has the leader's role in reducing emissions, increasing renewable energy usage and in the use of combined heat and power as well as in district heating. Moreover, the prices are very competitive and the operational reliability is on it's own level. In addition digitalisation allows the energy industry to be more efficient with lower expenses \parencite{Tekes:2017}.

According to the analyze made by \textcite{Deloitte} digitalisation offers opportunities in four different levels. First level is concentrating on improving efficiency and profitability, but on this level there is no new business. However, the first level is a good base for then proceed to the second level, which is concentrating in finding new businesses and improving existing ones. On the third level, the company is automating the business and they use data in their operations. Lastly, on the fourth level the company is innovating new models and they are superior in taking advantage from the digitalisation.

The relationship between the energy provider and the customer is also changing. In the future it is possible for the energy company to choose its role whether the company wants to expand their offering and to provide services for the customer and through that increase the relationship with them. It is also possible for the energy company to stay as they are today and being seen only as the energy provider. Since, the customers are seeking more possibilities in order to be more cost-efficient and to have more stable indoor conditions, it offers opportunities for new service providers and for energy companies as well as for the deputy landlords who are capable to answer the needs of the customers. \parencite{Deloitte} 

\subsection{Service Business in Energy Industry}

To deliver the value proposition and sell the service for customers the company needs to think the channels which are used to reach the customers. These channels offers touch points for creating customer experience and the channel phases can be divided into awareness, evaluation, purchase, delivery and after sales. In order to design the way to reach the customers it is important to think how they are reached now, what works and how do the company wants to reach them. \parencite{BusinessModelGeneration:2010} Customer journey maps offers a way to design such channel and are presented later in this chapter.

In a business to customer ecosystem \textcite{MarketingPlans:2016} list methods for marketing products. These methods are personal selling, direct mail and email, retail outlets and and brands. In personal selling the products are usually difficult to sell meaning that the value is difficult to explain, highly profitable and commodity items meaning that the differentiation is only happening for example during the sales process. Even though personal selling is an effective way to sell since the customer is engaged in the sales process itself, it is not a cheap way to make sales happen.

Direct mail and email allows the company to target specific types of customers with lower cost compared to personal selling even though the direct contact is made. Customers can receive for example offers and catalogues via direct mail and email. However, the downside of direct mail is that the response rate is low and some might feel they lose the personal contact. \parencite{MarketingPlans:2016}

In consumer products, it is normal that the suppliers have their offering distributed by retailers with whom they have good relations and who have decent organizations for retail because of the customers are dispersal. One of the most important thing in creating and maintaining relationships with customers is to have a strong brand. Brand allows for customers to match their values, aspirations and lifestyle with the brand and through this the company might not need that much of a personal contact with the customers since they are attracted to the brand itself. \parencite{MarketingPlans:2016}

In addition to these methods, the social media marketing can be used to increase the company's online visibility, strengthen the customer relationship and provide a platform for word of mouth advertising. \parencite{SocialMediaMarketing:2017}

\textcite{MarketingPlans:2016} highlights that the product promise is an important aspect in service marketing compared to product marketing because the value of the service is assessed only on consumption. So, buying a service requires trust and relationship between the customer and the supplier. In addition, evaluating an offer about service is hard and it requires tangible evidence of the quality.

In energy industry the whole market in district heating is around 2.3 Billion euros in the year 2013 according to the service business in the energy industry report done by \textcite{EnergyServiceBusiness:2015}. Additionally, only 25 Million euros of that market is for the energy efficiency and heating system maintenance services. This share of 25 Million euros is only about 1.2 percent from the market but it has been growing rapidly (approximately 30 \% per year) in the years 2009-2013. In the future, the sales of district heat is expected to decrease due to energy efficient new construction, renovations done in older buildings and gradual warming of the climate.

From the perspective of housing companies as customers, \textcite{Deloitte} described that housing companies are mostly interested in easily understandable solutions for the heating since they do not usually have professional understanding related to the heating. All in all the solutions should be lowering their overall costs and it was mentioned that the customers want to actually to see how it happens. Without these, the customers do not want to put any effort for the optimization of the heating. For the housing companies the motives affecting to heating were ranked as 1. Easiness 2. Price 3. Indoor condition 4. Reliability.

What it comes to the identified customer needs, \textcite{Deloitte} state that they need clear service information, which do not require any effort in using and producing added value. To answer this need, the energy companies should for example bring customer centricity into their service development, understand their customers and maintain their customer relationships better.

\section{Science Behind Purchase Decision Making}

For the people, like the researcher, who tend to think rationally the next concept might strike as something new.  However, it is important to understand how we humans make decisions and what affects the decision making in order to plan for example marketing campaigns which actually work. Thus, this chapter studies decision making and especially the ones happening on purchasing moments. Despite that this concept is concentrating on how individuals behave it should be considered relevant from the perspective of this study, since the people who makes the decisions in housing companies are behaving as individuals and not as a business unit.

Nobel Prize winner Daniel Kahneman has studied human decision making and in his book \citetitle{Kahneman:2011} he presents framework which divides human decision making and behaviour into two systems: \emph{System 1} and \emph{System 2}. System 1 is responsible for fast, automatic and intuitive actions which do not require effort and  there is no sense of voluntary control. Then again, System 2 allow us to make reflective and deliberate decisions which require thinking. To be more precise, operations of System 2 are easily disrupted when attention is drawn away.

System 1 and 2 are both active while we are awake, but System 2 is normally on so to speak low-effort mode. System 1 runs automatically and it can not be turned off, so for example if we see a word that is written in a language we know, we will read it. In addition, System 1 is making suggestions about impressions, intuitions, intentions and feelings. If those suggestions become endorsed by the System 2, they turn into beliefs and voluntarily actions. System 2 gets activated if System 1 runs into difficulty such as hard mathematical task for example 17 x 24. To enlighten more \textcite{Kahneman:2011} states that \emph{"System 2 is activated when an event is detected that violates the model of the world that System 1 maintains"}.

According to \textcite{Decoded:2013} it is crucial to understand how the previously presented systems work, since they determine all of our purchasing decisions regardless of brand, products, industries or categories. Normally, an average contact with an advertise is for example in mailing 2 seconds and in popular magazine 1.7 seconds. This highlights the fact that the core message in marketing communications should be delivered in a matter of seconds and that even the tiniest thing can influence the decision making.

Every signal we detect can have an affect on our decision, even a scent. This was proven in a test done in a shopping mall where people were exposed to different aromas. The ones who had been exposed to the scent of baking cookies were more likely to help an unknown person compared to the people who had not been exposed to that scent. Even though the people were not made specifically aware of the scent, it influenced their behaviour. This is called as a framing effect. \parencite{Decoded:2013}

The framing effect is one of the key concept in order to understand how decisions are made. The framing effect is explained by \textcite{Decoded:2013} with an example of two same colored grey squares where the background is different colored, as seen in the Figure~\ref{fig:squares}. The background of the inner square affect how we perceive the color of the inner square and that happens implicitly. Even though, we know the inner squares are same colored we do not see it and the perception is changed and through this it also changes our decision. Despite we have decided that the inner square is different color, we can not see it. System 1 and System 2 work together where the System 1 creates the frame and System 2 focuses on the detail. In marketing, the brand work as framing and it affects the customers' perceived value and willingness to pay for premium price even though the products would be identical.

\begin{figure}[ht]
  \begin{center}
    % here the width of the figure is set to 9 cm
    \includegraphics[scale=2, width=\textwidth]{dippa/images/squares.png}
    \caption{We see the inner grey squares in different colors.}
    \label{fig:squares}
  \end{center}
\end{figure}

Additionally, the System 1 is managing all of our perceptions, expectations, attitudes and motivational drivers which affect the purchase decisions we make even though those can not be noted. There is a huge potential in increasing the persuasiveness of marketing activities by understanding all of the opportunities how it is possible to affect on decision making. \parencite{Decoded:2013}

\textcite{Decoded:2013} states that the purchase decisions are based on purchase-pain relationship. This was proven in an experiment which focused on analyzing neural activity. In the study the researchers showed first picture of a brand/product then the same picture with price and finally the respondents needed to press yes or no button to show whether they would buy the product. First picture activated the so called \emph{reward system} in the brains which triggers when we see something we value. Seeing the second picture activated the same parts of the brain which activates when we experience pain. This indicated that price is not rational. The researcher then came to a conclusion that purchase decision happen if \emph{the relation between reward and pain exceeds a certain value}: the higher the difference (so to say net value) between the reward and the pain is, more likely we buy the product. So, to put it simple the purchase decision happens more likely when: \begin{center}\emph{reward - pain = high net value}\end{center}

According to \textcite{Decoded:2013} all the purchase decisions are based on the cost-value relationship, but it is notable that there are two sides to it: the implicit and the explicit processes which are based on the previously mentioned System 1 and System 2. The implicit process is affected by the context where the decision is made and it is sensitive to the past and habits. The explicit process, however, focuses on the real and objective facts about the product, thus it bases the decision making on reasoning. Since, the cost-value relationship has two sides, it is needed to describe them more precisely.

So, in order to increase the net value, \textcite{Decoded:2013} mentions that the perceived value needs to be maximized and the perceived cost needs to be reduced. The product's value is two folded: explicit value is what the customers expect the product to deliver and implicit value is the context around it which can be formed through brand, social context, packaging and associations. Even though, it was mentioned earlier that seeing price triggers the same parts in brain as pain, it can also influence the perceived value. However, this happens only when the price range is high in the specific category.

As with the value, price is also two folded and consists of the explicit and implicit perception. Explicit price is judged objectively, but it is possible to change the perceived cost with certain contextual signals. For example, showing the discounted price together with the original price is perceived cheaper than showing just the discounted price even though the prices are exactly the same. Thus, the implicit price is dependent on how the price is actually presented. In addition to money as a cost, behavioural costs are also important perspective to notice. They include the time and effort which are needed to gain the promised value, reward by either buying or consuming the product. Behavioural costs should be as low as possible, meaning that the ease of doing business with the company should be effective in order to optimize the customer's path to purchase. \parencite{Decoded:2013} These methods were exploited in the study in order to increase the net value and the practical description of the exploitation is in the chapter 3.4.

In addition \textcite{Sinek:2009} emphasises in his book \citetitle{Sinek:2009} that the company should first make clear \emph{why} the company or the product exists and only after that \emph{what} the product does. This is because the people do not buy what the company does, instead they buy why the company does it. The \emph{what} means in this context the product's features and rational benefits. It is important to explain the \emph{why}, because that drives the decision and \emph{what} the product does offers reasoning which will help to rationalize the appeal of the product.

The above mentioned principal is explained by \textcite{Sinek:2009} it is presented in the below:\\\\
\textbf{Starting with what}\\
We've got a new product.\newline
It pauses live TV.\newline
Skips commercials.\newline
Rewinds live TV.\newline
Memorizes your viewing habits and records shows on your behalf without your needing to set it.
\\\\
\textbf{Starting with why}\\
If you're the kind of person who likes to have total control of every aspect of your life, boy do we have a product for you.\newline
It pauses live TV.\newline
Skips commercials.\newline
Rewinds live TV.\newline
Memorizes your viewing habits and records shows on your behalf
without you needing to set it.\\\\
The first example is starting with \emph{what} and the second one with \emph{why}. The latter one first gives reasons on why the product is needed and only after that explains what the product is and what it does. So, it is easier for the people to rationalize the decision when they receive tangible proofs of the product after the \emph{why}. This approach was also utilized in the study and information of it can be found from the chapter 3.4.


\section{Customer Centricity}

\textcite{Drucker:2007} states in his book \emph{The Practice of Management} that the purpose of a business is to create customers by business actions. Moreover the customers are the ones who determines what a business is. For the business, it is most important to understand what the customers consider as value since it is them who decides what the business produces. In addition, according to \textcite{Parniangtong:2017} organizations' performance have also been explained with the focus on the outcomes which the customers value and are willing to pay. This means, that the customers are more satisfied and thus give a competitive advantage for the company. For a company to succeed in nowadays competition in the marketplace, customer centricity is a must \parencite{PathToCustomerCentricity:2006}.

Customer-centric thinking is built on four main arguments which are as follows:
\begin{enumerate}
\item Loyal customers should produce more money for a company.
\item New customers might not produce money immediately, but they should be profitable for the company in the future.
\item  It is more affordable to serve and introduce new offerings to long-term customers than to new customers.
\item  Long-term customers are more stable when it comes to pricing and a company's digressions, in addition they are more likely to spread word about the company. \parencite{Parniangtong:2017}
\end{enumerate}

Based on these arguments, the loyal customers bring more value to the company and through this, the competitive advantage of the company becomes better. Because customer-centric thinking drives from the goal where the customer equity is on high level, the company should focus on the customer relationship over lifetime. In order to achieve customer equity the company should know their customers, respond to their needs, appreciate them and maintain a profitable relationship with them over a long time. \parencite{Parniangtong:2017}

\textcite{Parniangtong:2017} states that customer value is created by a value chain which is customer driven, being closer with the customer, selling solution, having trust from the customers and offering intimacy with their customers through personalized experience. Having a customer-driven value chain compared to the traditional product oriented, the company can serve customers' needs better because then the customer needs will drive the change. This is different from the product oriented value chain where the customers' priorities are seen only at the end of the value chain and the company's mindset is revolving around how they can make more where they are good at. So, if the customers and their needs and priorities are in the center of attention, the company is more likely to produce products which will differentiate \parencite{Parniangtong:2017}.

Being closer with the client means that the company have customized their offerings to individual customers. This is reported to happen for example by collecting information about their customers in order to understand buying patterns and through this the company has been able to influence the future purchase. To do so, the company also needs to view the problems from the perspective of the client and not through their own products. When company does this, they focus on satisfying customer needs and solving their problems and thus it can be said that they sell solutions. \parencite{Parniangtong:2017}

The value of trust and intimacy go hand in hand. A company must gain trust from the customers in order to maintain the relationship. When trust is earned, the goal is to create value through customer intimacy because companies want that making business with them is easy. To do so, company must show to the customer that they care about them, value their business and are aware of their needs as individual and unique customers. \parencite{Parniangtong:2017}

The differences between product centricity and customer centricity is also discussed in the article wrote by \textcite{PathToCustomerCentricity:2006} where the comparison is made through the basic philosophy, business orientation, product positioning, organizational structure and focus, performance metrics, management criteria, selling approach and customer knowledge. The comparison is presented in the Figure~\ref{fig:productvscustomer} and it summarizes the differences between the two approaches.

\begin{figure}[ht]
  \begin{center}
    % here the width of the figure is set to 9 cm
    \includegraphics[scale=2, width=\textwidth]{dippa/images/ProductVsCustomer.png}
    \caption{Differencies in product-centric and customer-centric approaches by \textcite{PathToCustomerCentricity:2006}.}
    \label{fig:productvscustomer}
  \end{center}
\end{figure}

To summarise this chapter \textcite{Fader:2012} mentions that to gain the long-term profits from a customer centric view, the company need to succeed in the following areas: customer acquisition, customer retention and customer development. What this means is that the company should get highly committed customers to be as advocates for the company through customer acquisition, increase the amount of referrals and understand the cost as well as the value of new customer acquisition. The goal of customer retention is to lengthen the relationships between the company and the best customers as well as to lower the cost of maintaining such relationships. Lastly, customer development strives to develop customers in the direction where they would buy more products or services from the company.

\subsection{Ways to Bring Customers into Center of Attention}

In this section, the researcher present how service design can be facilitated in order to bring the customers into center of attention. The presented examples are chosen based on the relevancy of the actual research and the easiness to apply them in the study. In chapter 3, the researcher will describe how the customer segmentation was used in the study.

\subsubsection*{Customer Segmentation}

In the Lean Entrepreneur, written by \textcite{LeanEntrepreneur:2013}, the fact that the \emph{who} will buy is as important as the product. To serve for a specific group of people, segmentation is needed in order to define the marketing and sales plans as well as the distribution channels to answer the needs of that specific group. If there is no segmentation done, the company do not have an understanding on who is testing the value proposition. In addition, it is impossible in the beginning to create a value proposition that fits the whole market. Instead, it should be notified that successful solutions develop over time through the early adapters.

The Law of Diffusion of Innovations explains how the innovations spread through societies and even more, it explains how ideas spread. The Law of Diffusion of Innovations was first described by Everett M. Rogers and in the 1990's Geoffrey Moore went deeper and applied the principle to high-tech product marketing. Simon Sinek explains and refers to this Law of Diffusion in his book \citetitle{Sinek:2009}. According to the law the population is divided into five segments: innovators, early adopters, early majority, late majority and laggards. These segments and their percentages are presented in the Figure~\ref{fig:lawofdiffusion}.

As seen in the Figure~\ref{fig:lawofdiffusion} innovators are 2.5 percent of the population and early adopters are 13.5 percent of the population. According to \textcite{Sinek:2009} these two segments have a lot in common as they both appreciate the crafted advantages of the new ideas or technologies and they rely on their intuition and trust their gut with decisions. Additionally, early adapters recognize the value of a new idea and are not that sensitive for some minor imperfections because they understand and know the idea's or product's potential. It is important for them that they are the first ones to use the product. To make this more clear with an example, the segments who are on the left side of the diffusion curve are stated to be the ones who line up in front of the Apple store hours before the store opens in order to be among the first to buy a new iPhone.

\begin{figure}[ht]
  \begin{center}
    % here the width of the figure is set to 9 cm
    \includegraphics[scale=2, width=\textwidth]{dippa/images/lawofdiffusion.png}
    \caption{Segmentation of population according to the Law of Diffusion \parencite{Sinek:2009}.}
    \label{fig:lawofdiffusion}
  \end{center}
\end{figure}

Early majority together with the late majority forms 68 percent of the population. The laggards are 16 percent of the population and are described to by for example touchscreen phones only when they do not make phones which have buttons anymore. The early and late majority are described to be more practical in a sense that they evaluate the choices more rational. The difference between the early and late majority lies in the fact that the early majority is a bit more open towards new ideas and technologies compared to the late majority. The more right in the curve the population fall the less they believe what the company believe even though they might need what the company offers. It is stated that it is important to identify these segments because then it is possible to avoid doing business with them. \parencite{Sinek:2009}

\textcite{Sinek:2009} mentions also that if a company wants to achieve mass-market of some level, the Law of Diffusion must be considered. If a company tries to convince the early majority before it has appealed to the innovators and early adapters, it is seen as nearly mission impossible. This is because early majority do not want to try anything before someone else has tried it first and recommended it. So, the mass-market success can only be achieved if the company has penetrated first between 15 to 18 percent of the market which means that the company needs to focus on the early adopters and after that the rest of the market will follow.

Moreover there are many different things that suggest that customers belong to different segments. According to \textcite{LeanEntrepreneur:2013} and \textcite{BusinessModelGeneration:2010} customers belong to different segments if :
\begin{itemize}
\item They use different medias and hang out in different places, meaning that they can be reached through different distribution channels.
\item Their needs are clearly different thus it justify a distinct offer since they might be willing to pay for different things.
\item They expect different solutions and have different profitability.
\item Methods for selling are not the same.
\end{itemize}
The main takeaway from here is, that people might and will use the same product in different ways and it is as important to design how to market and sell the product as it is to design the actual product for their needs. And to do so, it is crucial that the company knows and understands the customers.

Another important aspect of customer segmentation is to design the value proposition. \textcite{BusinessModelGeneration:2010} states that value propositions describes the value for a customer through various elements which fill the needs of the customers. To help creating the value proposition these aspect can be for example newness of the need which is usually related to the technology, price meaning that the similar value is offered in a lower price which fills the needs of price-sensitive customers or cost which promises to reduce the costs with the help of the provided service. 

\subsubsection*{Customer Journey Maps}

Based on \textcite{Kalbach:2016} operational efficiency of a company is thought as more important than their customer satisfaction. To turn this situation around, alignment diagrams can be used to help organizations to think how their business fit to the lives of customers. Alignment diagrams present the interactions between a customer and an organization in a storytelling way. Additionally, the diagrams offer a way to show both sides of value creation since they help organizations to visualize the strategy as well as they help to design experiences. \parencite{Kalbach:2016}

Alignment diagrams are beneficial because they build empathy, provide a big picture, break silos in organizations, help focusing and point where improvements and innovations can be made. Understanding better people's thoughts and feelings through alignment diagrams build empathy which allows an organization to learn about their customers and real-world human conditions. With the alignment diagrams it is possible to share the understanding and provide a big picture in organizations because the actions become more consistent and the decision making is influenced by the diagrams. \parencite{Kalbach:2016}

Because the alignment diagrams allows to map the customer experiences across different organization departments, it is possible to break down silos. In addition they help organizations to focus because alignment diagrams match the outward-facing activities with inward-facing activities. Lastly, the alignment diagrams reveal opportunities for improvements and growth because the presentation of the information allows it to be understood without middlemen. \parencite{Kalbach:2016}

According to \textcite{Kalbach:2016} the value in mapping experiences becomes from the fact that it allows to discover innovative ways to make interactions better. Moreover, the whole design of the system is then easier to make coherent. Customer journey maps CJMs are type of alignment diagrams and those are described to be visualizations about a customer's experiences and usually they illustrate how customers make a choice for example a decision to buy a service. So, customer journey maps present the steps the customers take while engaging with the company through different touchpoints \parencite{Richardson:2010}.

More precisely, CJM's present the interactions in a chronological way including for example customer's actions, thoughts, feelings and pain points. From the organizational view point the CJM's present the roles and departments who are part of creating the experience for the customer. \parencite{Kalbach:2016} The practical usage of customer journey map is presented later in the chapter 3.4, where the research implementation is described.








